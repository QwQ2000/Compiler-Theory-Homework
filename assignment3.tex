\documentclass{article}
\usepackage{ctex}
\title{编译原理 作业3}
\author{} 
\date{}
\usepackage[a4paper,left=10mm,right=10mm,top=15mm,bottom=15mm]{geometry} 
\usepackage{graphicx} 
\usepackage{makecell}

\begin{document}
\section*{4.1}
\noindent 
\textbf{(1)}消去左递归得文法:\\
$S\rightarrow a|\^{}|(T)\\
T\rightarrow ST'\\
T'\rightarrow ,ST'|\epsilon$\\
则不带回溯的递归子程序如下所示:
\begin{tabbing}
    \hspace{0.5cm} \= \hspace{0.5cm}\= \kill
    PROCEDURE T';\\
    BEGIN\\
    \>S;T';\\
    END;\\ \\
    PROCEDURE T;\\
    BEGIN\\
    \>IF\ SYM=',' THEN\\
    \>BEGIN\\
    \> \>ADVANCE;\\
    \> \>S;T';\\
    \>END;\\
    END;\\\\
    PROCEDURE S;\\
    BEGIN\\
    \>IF SYM='a' OR SYM='$\^{}$' THEN\\
    \>\>ADVANCE;\\
    \>ELSE IF SYM='(' THEN\\
    \>BEGIN\\
    \>\>ADVANCE;\\
    \>\>T;\\
    \>\>IF SYM=')' THEN ADVANCE;\\
    \>\>ELSE ERROR;\\
    \>END;\\
    \>ELSE ERROR;\\
    END;
\end{tabbing}
\textbf{(2)}FIRST(S)=\{a,$\^{}$,(\}\\
FIRST(T)=\{a,$\^{}$,(\}\\
FIRST(T')=\{,\ ,$\epsilon$\}\\
FOLLOW(S)=\{\#,\ ,\ ,)\}\\
FOLLOW(T)=\{)\}\\
FOLLOW(T')=\{)\}\\
由于\textbf{(1)}已经消去了左递归,故文法不含左递归;考察S,T,T'的各产生式,其候选首符集显然不相交;\\
又FIRST(T')$\cap$FOLLOW(T')=$\Phi$.综上所述,文法为LL(1)文法.
预测分析表如下所示:
\begin{table}[h]
    \centering
\begin{tabular}{|p{2cm}<{\centering}|p{2cm}<{\centering}|p{2cm}<{\centering}|p{2cm}<{\centering}|p{2cm}<{\centering}|p{2cm}<{\centering}|p{2cm}<{\centering}|}   
    \hline
    \ & a & $\^{}$ & ( & ) & , & \# \\
    \hline
    S & S$\rightarrow$a & S$\rightarrow\^{}$ & S$\rightarrow$(T) & \ & \ & \ \\
    \hline
    T & T$\rightarrow$ST' & T$\rightarrow$ST' & T$\rightarrow$ST' & & & \\
    \hline
    T' & & & & T'$\rightarrow\epsilon$& T'$\rightarrow$,ST' & \\
    \hline
\end{tabular}
\end{table}
\\ \\
\section*{4.2}
\noindent 
\textbf{(1)}FIRST(E)=\{(,a,b,$\^{}$)\}\\
FIRST(E')=\{+,$\epsilon$\}\\
FIRST(T)=\{(,a,b,$\^{}$)\}\\
FIRST(T')=\{(,a,b,$\^{}$),$\epsilon$\}\\
FIRST(F)=\{(,a,b,$\^{}$)\}\\
FIRST(F')=\{*,$\epsilon$\}\\
FIRST(P)=\{(,a,b,$\^{}$\}\\
FOLLOW(E)=\{\#,)\}\\
FOLLOW(E')=\{\#,)\}\\
FOLLOW(T)=\{\#,+,)\}\\
FOLLOW(T')=\{\#,+,)\}\\
FOLLOW(F)=\{(,a,b,$\^{}$,),+,\#\}\\
FOLLOW(F')=\{(,a,b,$\^{}$,),+,\#\}\\
FOLLOW(P)=\{(,a,b,$\^{}$,),+,*,\#\}\\
\textbf{(2)}
观察易得,此文法不含有左递归.\\
FIRST(+E)$\cap$FIRST($\epsilon$)=$\Phi$\\
FIRST(T)$\cap$FIRST($\epsilon$)=$\Phi$\\
FIRST(*F')$\cap$FIRST($\epsilon$)=$\Phi$\\
FIRST((E))$\cap$FIRST(a)$\cap$FIRST(b)$\cap$FIRST($\^{}$)=$\Phi$
FOLLOW(E')$\cap$FIRST(E')=$\Phi$\\
FOLLOW(T')$\cap$FIRST(T')=$\Phi$\\
FOLLOW(F')$\cap$FIRST(F')=$\Phi$\\
综上所述,此文法为LL(1)文法.\\
\textbf{(3)}
\begin{table}[h]
    \centering
\begin{tabular}{|p{1.4cm}<{\centering}|p{1.4cm}<{\centering}|p{1.4cm}<{\centering}|p{1.4cm}<{\centering}|p{1.4cm}<{\centering}|p{1.4cm}<{\centering}|p{1.4cm}<{\centering}|p{1.4cm}<{\centering}|p{1.4cm}<{\centering}|}   
    \hline
    & + & * & ( & ) & a & b & $\^{}$ & \# \\
    \hline
    E &&& E$\rightarrow$TE' && E$\rightarrow$TE' & E$\rightarrow$TE' & E$\rightarrow$TE' &\\
    \hline
    E' & E'$\rightarrow$+E &&& E'$\rightarrow\epsilon$ &&&& E'$\rightarrow\epsilon$ \\
    \hline
    T &&& T$\rightarrow$FT' && T$\rightarrow$FT' & T$\rightarrow$FT' & T$\rightarrow$FT' &\\
    \hline
    T' & T'$\rightarrow\epsilon$ & & T'$\rightarrow$T & T'$\rightarrow\epsilon$ & T'$\rightarrow$T & T'$\rightarrow$T & T'$\rightarrow$T & T'$\rightarrow\epsilon$ \\
    \hline
    F &&& F$\rightarrow$PF' && F$\rightarrow$PF' & F$\rightarrow$PF' & F$\rightarrow$PF' & \\
    \hline
    F' & F'$\rightarrow\epsilon$ & F'$\rightarrow$*F' & F'$\rightarrow\epsilon$ & F'$\rightarrow\epsilon$ & F'$\rightarrow\epsilon$ & F'$\rightarrow\epsilon$ & F'$\rightarrow\epsilon$ & F'$\rightarrow\epsilon$ \\
    \hline
    P &&& P$\rightarrow$(E) && P$\rightarrow$a & P$\rightarrow$ b & P$\rightarrow$$\^{}$ &\\
    \hline
\end{tabular}
\end{table}
\\
\textbf{(4)}
\begin{tabbing}
    \hspace{0.5cm} \= \hspace{0.5cm}\= \hspace{0.5cm}\= \kill
    PROCEDURE E;\\
    BEGIN\\
    \>IF SYM='(' OR SYM='a' OR SYM='b' OR SYM='$\^{}$' THEN \\
    \>BEGIN\\
    \>\>T;E';\\
    \>END;\\
    \>ELSE ERROR;\\
    END;\\\\

    PROCEDURE E';\\
    BEGIN\\
    \>IF SYM='+' THEN\\
    \>BEGIN\\
    \>\>ADVANCE;E;\\
    \>END;\\
    \>ELSE IF SYM!=')' AND SYM!='\#' THEN ERROR;\\
    END;\\\\

    PROCEDURE T;\\
    BEGIN\\
    \>IF SYM='(' OR SYM='a' OR SYM='b' OR SYM='$\^{}$' THEN\\
    \>BEGIN\\
    \>\>F;T';\\
    \>END;\\ 
    \>ELSE ERROR;
    END;\\\\

    PROCEDURE T';\\
    BEGIN\\
    \>IF SYM='(' OR SYM='a' OR SYM='b' OR SYM='$\^{}$' THEN T;\\
    ELSE IF SYM='*' THEN ERROR;\\
    END;\\\\

    PROCEDURE F;\\
    BEGIN\\
    \>IF SYM='(' OR SYM='a' OR SYM='b' OR SYM='$\^{}$' THEN\\
    \>BEGIN\\
    \>\>P;F';\\
    \>END;\\ 
    \>ELSE ERROR;
    END;\\\\

    PROCEDURE F';\\
    BEGIN\\
    \>IF SYM='*' THEN\\
    \>BEGIN\\
    \>\>ADVANCE;F';\\
    \>END;\\
    END;\\\\

    PROCEDURE P;\\
    BEGIN\\
    \>IF SYM='a' OR SYM='b' OR SYM='$\^{}$' THEN ADVANCE;\\
    \>ELSE IF SYM='(' THEN\\
    \>\>ADVANCE;E;\\
    \>\>IF SYM=')' THEN ADVANCE;\\
    \>\>ELSE ERROR;\\
    \>END;\\
    \>ELSE ERROR;\\
    END;
\end{tabbing}
\section*{4.3}
\noindent 
\textbf{(1)}是LL(1)文法,满足LL(1)文法的三个条件.\\
\textbf{(2)}不是LL(1)文法,A违反了条件3.\\
\textbf{(3)}不是LL(1)文法,A和B违反了条件3.\\
\textbf{(4)}是LL(1)文法,满足LL(1)文法的三个条件.
\section*{4.4}
\noindent 
\textbf{(1)}FIRST(Expr)=\{-,(,id\}\\
FIRST(ExprTail)=\{-,$\epsilon$\}\\
FIRST(Var)=\{id\}\\
FIRST(VarTail)=\{(,$\epsilon$\}\\
FOLLOW(Expr)=\{\#,)\}\\
FOLLOW(ExprTail)=\{\#,)\}\\
FOLLOW(Var)=\{-,\#,)\}\\
FOLLOW(VarTail)=\{-,\#,)\}\\
则LL(1)分析表如下所示:
\begin{table}[h]
    \centering
    \begin{tabular}{|p{2.6cm}<{\centering}|p{2.6cm}<{\centering}|p{2.6cm}<{\centering}|p{2.6cm}<{\centering}|p{2.6cm}<{\centering}|p{2.6cm}<{\centering}|}
        \hline
        & - & id & ( & ) & \# \\
        \hline
        Expr& Expr$\rightarrow$-Expr & Expr$\rightarrow$Var ExprTail & Expr$\rightarrow$(Expr) && \\
        \hline
        ExprTail & ExprTail$\rightarrow$-Expr &&& ExprTail$\rightarrow\epsilon$ & ExprTail$\rightarrow\epsilon$ \\
        \hline
        Var && Var$\rightarrow$id VarTail &&& \\
        \hline
        VarTail & VarTail$\rightarrow\epsilon$ && VarTail$\rightarrow$(Expr) & VarTail$\rightarrow\epsilon$ & \\
        \hline
    \end{tabular}
\end{table}
\\
\textbf{(2)}
\begin{tabbing}
    \hspace{1.5cm} \= \hspace{7cm}\= \hspace{3cm}\= \hspace{3cm}\= \kill
    \textbf{步骤}\>\textbf{符号栈}\>\textbf{输入串}\>\textbf{所用产生式}\\
    0\>\#Expr\>id-\ -id((id))\#\\
    1\>\#ExprTail Var\>id-\ -id((id))\#\>Expr$\rightarrow$Var ExprTail\\
    2\>\#ExprTail VarTail id\>id-\ -id((id))\#\>Var$\rightarrow$id VarTail\\
    3\>\#ExprTail VarTail\>-\ -id((id))\#\>\\
    4\>\#ExprTail\>-\ -id((id))\#\>VarTail$\rightarrow\epsilon$\\
    5\>\#Expr - \>-\ -id((id))\#\>ExprTail$\rightarrow$-Expr\\
    6\>\#Expr\>-id((id))\#\\
    7\>\#Expr -\>-id((id))\#\>Expr$\rightarrow$-Expr\\
    8\>\#Expr\>id((id))\#\>\\
    9\>\#ExprTail Var\>id((id))\#\>Expr$\rightarrow$Var ExprTail\\
    10\>\#ExprTail VarTail id\>id((id))\#\>Var$\rightarrow$id VarTail\\
    11\>\#ExprTail VarTail\>((id))\#\\
    12\>\#ExprTail ) Expr (\>((id))\#\>VarTail$\rightarrow$(Expr)\\
    13\>\#ExprTail ) Expr\>(id))\#\\
    14\>\#ExprTail ) ) Expr (\>(id))\#\>Expr$\rightarrow$(Expr)\\
    15\>\#ExprTail ) ) Expr\>id))\#\\
    16\>\#ExprTail ) ) ExprTail Var\>id))\#\>Expr$\rightarrow$Var ExprTail\\
    17\>\#ExprTail ) ) ExprTail VarTail id\>id))\#\>Var$\rightarrow$id VarTail\\
    18\>\#ExprTail ) ) ExprTail VarTail\>))\\
    19\>\#ExprTail ) ) ExprTail\>))\#\>VarTail$\rightarrow\epsilon$\\
    20\>\#ExprTail ) )\>))\#\>ExprTail$\rightarrow\epsilon$\\
    21\>\#ExprTail )\>)\#\\
    22\>\#ExprTail\>\#\\
    23\>\#\>\#\>ExprTail$\rightarrow\epsilon$\\
\end{tabbing}
\end{document}
